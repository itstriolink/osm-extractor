\chapter{Project Management}
\section{Organization}
The work for this project has been done mainly by me (coding completely by me), with the help of my supervisor: \href{mailto:stefan.keller@ost.ch}{Prof. Stefan Keller}
and with technical help from the OpenRefine Dev community.\\
\newline
The extensions were also tested after each release by \href{mailto:felix.reiniger@ost.ch}{Felix Reiniger} using the testing
workflow described in the \hyperref[sec:test-workflow]{Testing chapter} of the Appendices.
\newline
The extensions are open-source and licensed under the \href{https://mit-license.org/}{MIT License}.
\newline
The source code/repositories for the extensions can be found here:
\begin{enumerate}
    \item \href{https://gitlab.com/labiangashi/osm-extractor}{https://gitlab.com/labiangashi/osm-extractor}
    \item \href{https://gitlab.com/labiangashi/geojson-export}{https://gitlab.com/labiangashi/geojson-export}
\end{enumerate}
\newpage
\section{Planning and Coordination}
Only the initial meeting for the thesis was held physically with me and my suppervisor in Rapperswil,
in the OST building, in order to discuss the initial steps and requirements for the thesis.\\
The meetings/stand-ups were thereafter held online because of distance but also because of the COVID-19 pandemic, they consisted of me and my supervisor as participants.\\
An agile software development process was used and each sprint lasted approx. 2 weeks, with some requirements changing in-between sprints,
some of them shifting and some of them being removed completely (or added). \\
\newline
The initial sprints consisted of mainly deciding the technical requirements for the extensions, the scope of the project etc.
After that, began the phase of developing the extensions, where requirements were sometimes altered, and the last phase was of course, documenting the project.
\pagebreak
