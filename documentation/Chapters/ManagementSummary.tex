\addtocontents{toc}{\protect\setcounter{tocdepth}{0}}
\addtotoc{Management Summary}
\chapter*{Management Summary}
In this project, we try to incorporate spatial data representations in OpenRefine as defined by the \href{https://www.ogc.org/standards/sfa}{OGC Simple Feature Access} geometry types, and by encoding them as a \href{https://www.ogc.org/standards/wkt-crs}{Well-Known Text (WKT)} or as columns containing latitude/longitude coordinates. These extensions extend OpenRefine to import OpenStreetMap data, to process spatial data, and to export this data in the GeoJSON format.
Two extensions are built for achieving such goals.\\
\newline
The thesis also includes writing a worksheet/article that explains and demonstrates how to use the OpenRefine application in the means of exploring, cleaning and integrating data with it.
\section*{Goals}
\subsection*{The OSM Extractor Extension}
For the OSM Extractor extension, these are the goals:
\begin{enumerate}
    \item Build an OpenRefine extension using OpenRefine\textquotesingle s extension architecture
    \item The extension must be able to utilize the Overpass API and other different OpenStreetMap Java libraries to successfully
    integrate OpenStreetMap data into OpenRefine
    \item The OpenStreetMap elements must be converted to \text{Simple Feature} geometry before being inserted to OpenRefine
    \item The extension\textquotesingle s forms and dialogs must be user-friendly and the process of integrating data must be as efficient
    and as least-effort as possible (for the user)
\end{enumerate}
\subsection*{The GeoJSON Export Extension}
For the GeoJSON Export extension, these are the goals:
\begin{enumerate}
    \item Build an OpenRefine extension using OpenRefine\textquotesingle s extension architecture
    \item The extension must be able to export data of an OpenRefine project into the GeoJSON (.geojson) format. Latitude/Longitude
    can be used for exporting \mintinline{bash}{Points} and WKT strings can be used for exporting any type of \textit{Simple Feature} geometry.
    A combination of both can be used for exporting
    \item The extension\textquotesingle s forms and dialogs must be user-friendly and the process of exporting data must be as efficient
    and as least-effort as possible (for the user)
\end{enumerate}
\section*{Results}
\subsection*{The OSM Extractor Extension}
The OSM Extractor extension has been
successfully implemented and is able to fully integrate
OSM data represented as a geometry of type Point,
LineString, MultiLineString or MultiPolygon. OSM tags
(attributes) are also included whereas main tags are
sorted to the beginning. It also includes a GREL
function named "interiorPoint()" that extracts the
center point of a geometry, this allows for easier
integration e.g. with
Wikidata.
\begin{figure}[H]
        \includegraphics[width=\textwidth]{./Figures/ManagementSummary/osm_extractor_overpass_query}
        \caption{OpenStreetMap data for all the drinking water amenities in Zürich, Switzerland, taken from \href{http://overpass-turbo.osm.ch/}{Overpass Turbo CH}}
\end{figure}
\begin{figure}[H]
    \includegraphics[width=\textwidth]{./Figures/ManagementSummary/osm_extractor_project}
    \caption{OpenRefine project created using the OSM Extractor extension, containing geometry objects for all the drinking water amenities in Zürich, Switzerland}
\end{figure}
\subsection*{The GeoJSON Export Extension}
The extension has been successfully developed using OpenRefine\textquotesingle s extension architecture and is able to create a GeoJSON file while letting the
user choose the columns for geometry and columns for geometry properties. In
addition, the decimal digits of the coordinates can be
overridden.
\begin{figure}[H]
    \includegraphics[width=\textwidth]{./Figures/ManagementSummary/osm_extractor_project}
    \caption{An OpenRefine project containing data for all the drinking water amenities in Zürich, Switzerland}
\end{figure}
And here is a portion of the GeoJSON exported file contents from the same project:
\begin{minted}[breaklines, tabsize=2, linenos]{json}
{
  "type" : "FeatureCollection",
  "features" : [ {
    "type" : "Feature",
    "geometry" : {
      "type" : "Point",
      "coordinates" : [ 8.718769, 47.350209 ]
    },
    "properties" : {
      "@timestamp" : "2021-07-06T21:02:59Z",
      "amenity" : "drinking_water",
      "@user" : "Dorni4",
      "@version" : "1",
      "@changeset" : "107521343",
      "@id" : "8899069808",
      "@uid" : "5619448",
      "@visible" : "true"
    }
  }, {
    "type" : "Feature",
    "geometry" : {
      "type" : "Point",
      "coordinates" : [ 8.603137, 47.536676 ]
    },
    "properties" : {
      "@timestamp" : "2013-01-01T19:40:39Z",
      "amenity" : "drinking_water",
      "@user" : "keenonkites",
      "@version" : "1",
      "@changeset" : "14491114",
      "@id" : "2091396746",
      "@uid" : "492934",
      "@visible" : "true"
    }
  }, {
    "type" : "Feature",
    "geometry" : {
      "type" : "Point",
      "coordinates" : [ 8.58352, 47.358378 ]
    },
    "properties" : {
      "@timestamp" : "2018-02-12T14:45:47Z",
      "amenity" : "drinking_water",
      "@user" : "d_berger",
      "@version" : "5",
      "@changeset" : "56296709",
      "@id" : "698059005",
      "@uid" : "1234273",
      "@visible" : "true"
    }
  } ]
}
\end{minted}
As seen in the JSON snippet above, a \mintinline{bash}{FeatureCollection} GeoJSON object is created containing all of the
\mintinline{bash}{Points} from the OpenRefine project, along with their non-spatial attributes (\textit{@timestamp}, \textit{@id}, \textit{amenity} etc.)
\section*{Future Outlook}
Contributions to the OpenRefine core project are also in the scope. One of the items discussed with the OpenRefine Dev Community is collaborating with them to continue implementing a feature that allow users to add descriptions and different custom metadata to their OpenRefine project\textquotesingle s columns. These works would go directly into the core OpenRefine project. The discussion can be viewed here: \href{https://groups.google.com/g/openrefine-dev/c/xWoWiAA6KhQ}{https://groups.google.com/g/openrefine-dev/c/xWoWiAA6KhQ}
\subsection*{The OSM Extractor Extension}
Future versions of OSM Extractor could save the
Overpass query, similar to a saved database connection (connection string). Or it
could allow synchronization to keep an OpenRefine
project up-to-date with OSM. The user would need a \textit{Sync} button on the OpenRefine project which would sync the current data with the latest updates/additions on the OpenStreetMap database.
\subsection*{The GeoJSON Export Extension}
Future versions of this extension could consider adding the  \mintinline{bash}{GeometryCollection} Geometry object
into the supported GeoJSON geometry types. All the other GeoJSON Geometry types are supported in the current version.