\chapter{The Worksheet on OpenRefine}
As part of the thesis work, one of the tasks is to write a article-like worksheet which explains and demonstrates how to explore, clean and integrate data with OpenRefine.\\
\newline
The article is written using a mark-down editor platform called \href{https://md.coredump.ch/}{HedgeDoc}. It has many easy-to-use features for creating articles and pages that are responsive and can be easily and quickly shared with the collaborators and the world.\\
\newline
The worksheet starts off by giving an introduction on Data Integration and Data Enrichment. It then continues by explaining OpenRefine basics, including it\textquotesingle s GUI. d. \\
\newline
After the users gain a basic understanding of OpenRefine, some of its functionality features are explained (and demonstrated in the form of exercises later in the worksheet):
\begin{itemize}
	\item Importing and exploring data
	\item Transforming data
	\item Joining datasets from different sources
	\item Validation and deduplication of data
	\item Exporting data
\end{itemize}
The worksheet then gives an introduction to geocoding and how to enrich data using geocoding. An exercise is included in the geocoding chapter so that the users gain a better understanding of the concept. The last functionality explained is web scraping, which also includes an exercise on how to scrape web data using OpenRefine.\\
\newline
The entire worksheet can be found in the following link: \href{https://md.coredump.ch/s/ICjanZtTG}{https://md.coredump.ch/s/ICjanZtTG} and also in the appendices section of this paper.