\chapter{Architecture and Design}
\section{Architecture}
The extensions are built using the \href{https://www.java.com/en/}{Java} programming language (Java Platform SE 8),
which is fully compliant with the OpenRefine development environment.
The extension are built as a "separate" component from OpenRefine, meaning it has to manually be added/installed to an
existing OpenRefine application environment.\\
\newline
The extension is compiled using \href{https://maven.apache.org/index.html}{Apache Maven} and consists of client-side
resources such as .html, .css, .less, .js files and of server-side resources i.e. Java classes. These Java classes are
bundled during the initialization of OpenRefine and can consist of controllers, commands, functions etc. They are invoked
using HTTP requests such as GET or POST with AJAX. All of the necessary client-side resources and server-side resources must be
injected to OpenRefine using the \mintinline{java}{init()} function of the extension.\\
\newline
When the extension is built, all of the Java classes and the client-side resources are stored in a directory which need to be copied to
the OpenRefine installation \\extension directory in order for it to be bundled together with the OpenRefine application.\\
\newline
An extension of OpenRefine has the following file structure:
\begin{minted}
[
    frame=lines
]
{bash}
pom.xml
  src/
      com/foo/bar/... *.java source files
  module/
      *.html, *.vt files
      scripts/... *.js files
      styles/... *.css and *.less files
      images/... image files
      MOD-INF/
          lib/*.jar files
          classes/... java class files
          module.properties
          controller.js
\end{minted}
More information on installing extensions can be found \href{https://docs.openrefine.org/manual/installing#installing-extensions}{here}.
OpenRefine's official documentation also includes an article on how to write extensions \href{https://docs.openrefine.org/technical-reference/writing-extensions}{here}.\\
\newline
\subsection{The Solution Stack}
This extension was built using the Java programming language, version 8, for the back end. For the front-end portion,
a combination of Javascript, the jQuery library, LESS and HTML was used. \\
\newline
\href{https://www.java.com/en/}{Java} is a powerful general-purpose programming language.
It is used to develop desktop and mobile applications,
big data processing, embedded systems, and so on.
According to Oracle, the company that owns Java, Java runs on 3 billion devices worldwide, which makes Java one of the
most popular programming languages.~\cite{WhatIsJava}\\
\newline
The reason why this plugin was chosen to be built and developed using Java is
because of OpenRefine, it is programmed using Java,
so all its extensions must also be written in Java (using OpenRefine\textquotesingle s extension
architecture). All of the OpenRefine extensions are developed inside the
OpenRefine environment as a separate bundled directory and must use OpenRefine\textquotesingle s
native Java methods in order to be communicate with OpenRefine\textquotesingle s core engine.\\
It is also an Object-Oriented programming language, which runs under any operating system
(because of its compiler) and uses inheritance and abstract methods (Object Oriented programming principles),
which turned out to be useful and appropriate for the nature of this extension.\\
\newline

For the front-end, the usual front-end stack was used to develop the UI parts i.e. Javascript, jQuery, LESS and HTML.
\pagebreak
\section{Design Decisions}
This section of architecture and design contains information on the various decisions that were taken for this project and the reasons that stand behind those decisions.
There are decisions that belong to the project as a whole and also to each of the extensions individually.
\newline
An important notation for this chapter is the "Architectural Knowledge Management (AKM): "Y-Approach"~\cite{AKMHomepage} - In deciding how much design rationale is enough, the Y-approach has been followed~\cite{YApproach}. It helps capturing a decision by providing a skeleton of statements. One example of this approach is the following statement:
\begin{itemize}
	\item Architectural Decision \#1 - In the context/decision on how to efficiently fetch OpenStreetMap data, we decided to use The Overpass API as a service for communicating the OpenStreetMap back to our requests. This approach is compatible with the KISS principle, which was also applied in scope of this work.
\end{itemize}
The following sections of this chapter contain tables containing information on the decisions and their reasons for each of the extensions and also some other general decisions.
\pagebreak
\subsection{OSM Extractor}
\begin{table}[htbp]
    \centering
    \begin{tabularx}{\textwidth}{| X | X |} \hline
        \textbf{Decision} & \textbf{Reason}\\[1cm] \hline
        The \href{https://github.com/topobyte/osm4j}{osm4j} library is used for unmarshalling the OpenStreetMap XML data &
        The \href{https://github.com/topobyte/osm4j}{osm4j} library provides geometry and utility functions that are
        useful for processing OpenStreetMap XML data and for querying the Overpass API,
        turning it out to be a perfect fit for this extension\textquotesingle s implementation.\\ \hline
        The \href{https://github.com/locationtech/jts}{JTS Topology Suite} library is used to process the geometry objects. &
        OpenStreetMap data is first converted to \textit{Simple Feature} geometry objects before inserting into OpenRefine (explained later in the document).
        The \href{https://github.com/locationtech/jts}{JTS Topology Suite} provides useful classes and functions for dealing with such geometry objects (Points, LineStrings, Polygons etc).\\ \hline
        The Well-Known Text (WKT) geometry representation format is used to represent the \textit{Simple Feature} geometry objects. &
        Any type of \textit{Simple Feature} geometry object can be represented using WKT and it can be stored in a single
        cell without having any dependencies to another cell or needing more than one cell.\\ \hline
        \mintinline{bash}{Polygons} are not offered as a type of Geometry in the extension. &
        It is left up to the user to choose what geometries are Polygons and what still remain as MultiPolygons.\\ \hline
        \mintinline{bash}{out center;} is not supported as an Overpass QL command. &
        The reason is because the \href{https://github.com/topobyte/osm4j}{osm4j} library is a library that processes OpenStreetMap
        XML data and not precisely data retrieved from the Overpass API.
        Since the \mintinline{xml}{<center>} attribute is more Overpass specific, the osm4j repository does not support such an attribute.
        Here is also the issue opened in the osm4j repository: \href{https://github.com/topobyte/osm4j/issues/3}{https://github.com/topobyte/osm4j/issues/3}
        The interior point can be however calculated using the GREL function \mintinline{java}{interiorPoint()} that is included in this extension.\\ \hline
        \mintinline{sql}{[out:json];} and \mintinline{sql}{[out:csv];} are not supported in the extension. & This extension deals with only OpenStreetMap
        XML data and not with other types of representations. The \href{https://github.com/topobyte/osm4j}{osm4j} library also does not support reading
        \textit{JSON} or \textit{CSV} OpenStreetMap data.\\ \hline
    \end{tabularx}
    \caption{Design Decisions - OSM Extractor}
\end{table}
\pagebreak
\subsection{GeoJSON Export}
\begin{table}[htbp]
    \centering
    \begin{tabularx}{\textwidth}{| X | X |} \hline
        \textbf{Decision} & \textbf{Reason}\\[1cm] \hline
        The \href{https://github.com/bjornharrtell/jts2geojson}{jts2geojson} library is used for
        converting JTS geometry to GeoJSON. & The \href{https://github.com/bjornharrtell/jts2geojson}{jts2geojson} library is
        able to convert JTS geometries to GeoJSON and back. \\ \hline
        The \href{https://github.com/locationtech/jts}{JTS Topology Suite} library is used to process the geometry objects. &
        The \href{https://github.com/bjornharrtell/jts2geojson}{jts2geojson} requires JTS Geometry in order to be able to convert it to GeoJSON.
        The latitude/longitude columns and WKT strings are used together with the JTS library to create JTS Geometry.
        Points are created using their X and Y values into a JTS Point Geometry and WKT strings are converted to JTS geometry
        using the \href{https://locationtech.github.io/jts/javadoc/org/locationtech/jts/io/WKTReader.html}{\mintinline{bash}{WKT Reader}}
        JTS class. These geometries are then continously stored as GeoJSON features and at the end of the iteration,
        they are added to the \mintinline{bash}{FeatureCollection}.
        This \mintinline{bash}{FeatureCollection} is used as GeoJSON data for the exported GeoJSON file.\\ \hline
        The Well-Known Text (WKT) geometry is used as the geometry representation format for exporting to GeoJSON. &
        Any type of geometry object can be represented using WKT and it can be stored in a single
        cell without having any dependencies to another cell or needing more than one cell. \\ \hline
    \end{tabularx}
    \caption{Design Decisions - GeoJSON Export}
\end{table}
\pagebreak
\subsection{Other General Decisions}
\begin{table}[htbp]
    \centering
    \begin{tabularx}{\textwidth}{| X | X |} \hline
        \textbf{Decision} & \textbf{Reason}\\[1cm] \hline
        A new OpenRefine base data type for representing \textit{Geometry} objects is not added. & The reason for this is
        because OpenRefine\textquotesingle s core engine does not support the addition of adding a new base data type via an extension. The only way
        to add a new base data type would be to create an OpenRefine distribution and that wasn\textquotesingle t the goal of this project. \\ \hline
    \end{tabularx}
    \caption{Design Decisions - Other}
\end{table}
\pagebreak