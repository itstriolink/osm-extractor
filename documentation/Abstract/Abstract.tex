The need to integrate geospatial data into large data-driven applications has increased
rapidly over the last years. This automatically indicates to us that the geospatial data usage and popularity has also increased. A lot of non-profit organizations and consortiums around the globe continuously contribute to creating open standard data formats for representing geometry objects in order to standardize the access and exchange of such data. Along with the demand of integrating geospatial data into large data-driven applications, new tools have also emerged.\\
\newline
\href{https://openrefine.org/}{OpenRefine} is a low code development platform and a data integration tool for working with potentially messy data.
It provides various features for data transformation and cleaning, including the blending of data with web services and external data. It also includes its own scripting language, called the \href{https://docs.openrefine.org/manual/grel}{"General Refine Expression Language" (GREL)} - hence a low code development platform - which allows users to manage/transform their data with functions such as string processing, HTML parsing, JSON parsing etc. OpenRefine also has an extension architecture that allows users to install new extensions, and developers to add functionality using Java. Currently, OpenRefine cannot process geodata and does not support services for accessing geodata sources such as \href{https://www.openstreetmap.org/}{OpenStreetMap (OSM)}. \\
\newline
The idea behind this project is to create two OpenRefine extensions using OpenRefine\textquotesingle s extension architecture in order to incorporate
different spatial data representations in OpenRefine as defined by the \href{https://www.ogc.org/standards/sfa}{OGC Simple Feature Access} geometry types, and by encoding them as a \href{https://www.ogc.org/standards/wkt-crs}{Well-Known Text (WKT)} or as columns containing latitude/longitude coordinates. These extensions extend OpenRefine to import OpenStreetMap data, to process spatial data, and to export this data in the GeoJSON format. The extensions utilize many geospatial libraries and open geospatial data formats (or representations) in order to deliver compliant and useful features to the users.\\
\newline
The first extension, called "OSM Extractor", integrates OSM data represented as a geometry of type Point, LineString, MultiLineString or MultiPolygon. OSM tags (attributes) are also included whereas main tags are sorted to the beginning. It also includes a GREL function named \mintinline{java}{interiorPoint()} that extracts the center point of a geometry, this allows for easier integration e.g. with \href{https://www.wikidata.org/wiki/Wikidata:Main_Page}{Wikidata}.\\
\newline
The second extension, called "GeoJSON Export", creates a GeoJSON file while letting the user choose the geometry columns (of type latitude/longitude or WKT) and the non-spatial attributes of those objects. In addition, the decimal digits of the coordinates can be overridden.
\\
\newline
Keywords: \textit{OpenRefine, OpenStreetMap, Overpass API, GeoJSON, Simple Features, geospatial data, latitude, longitude, WKT}.