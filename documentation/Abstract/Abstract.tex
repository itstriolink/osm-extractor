The need of supporting and integrating geospatial data into large applications that deal with data has increased rapidly over the last years. This fact automatically indicates to us that the geospatial data usage and popularity has also increased along with its need.
A lot of non-profit organizations and consortiums around the globe continuously contribute to and support the creation of various geospatial technologies for simplifying the access and usage of geospatial data and location information. They also contribute to creating data formats for representing geometry objects in order to standardize the access and exchange of such data.
\newline
OpenRefine is a low code development platform and a tool for working with messy data.
It provides various features for
data transformation and cleaning, including the extension of data with web services and external data.
It also has an extension architecture that allows developers to add their own extensions to an existing OpenRefine application.
Extensions can extend the functionality of OpenRefine by offering new import options, new functions, new export formats etc.\\
\newline
The idea behind this project is to create two OpenRefine extensions using OpenRefine\textquotesingle s extension architecture in order to incorporate
different spatial data representations in OpenRefine by the example of providing the users with the feature of exporting their data to the GeoJSON format
and also by offering a new import Option in OpenRefine that allows the users to integrate OpenStreetMap data directly into a new OpenRefine project.\\
\newline
The extensions utilizes many geospatial libraries around the net and open geospatial data formats (or representations) in order to deliver useful and efficient features to the users.\\
\newline

Keywords: \textit{OpenRefine, OpenStreetMap, Overpass API, GeoJSON, Simple Features, geospatial data, latitude, longitude, WKT}